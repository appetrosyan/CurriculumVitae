\documentclass{CurriculumVitae}[10pt, condensed]
\usepackage{multicol}
\usepackage{enumitem}
\usepackage[square,numbers]{natbib}
\bibliographystyle{abbrvnat}
\usepackage{background}
\usepackage{pgfornament}
\usepackage{comment}
\usetikzlibrary{calc}
\usepackage[super]{nth}


\backgroundsetup{
  scale=1,
  opacity=1,
  angle=0,
  color=black,
  contents={%
    \begin{tikzpicture}[every node/.style={inner sep=0pt}]
      \node[anchor=north west](CNW)
      at (current page.north west) {\pgfornament[width=1.75cm]{61}};
      \node[anchor=north east](CNE)
      at (current page.north east) {\pgfornament[width=1.75cm,symmetry=v]{61}};
      \node[anchor=south west](CSW)
      at (current page.south west) {\pgfornament[width=1.75cm,symmetry=h]{61}};
      \node[anchor=south east](CSE)
      at (current page.south east) {\pgfornament[width=1.75cm,symmetry=c]{61}};
    \end{tikzpicture}%
  }
}


% Override this to blank if you don't want to have a coverletter.
\renewcommand{\coverletter} {%
%
  I'm the CEO of the Greybeard Consulting. Over the past few years I
  worked as a full-time Consultant for {Eclipse Laboratories Inc.} and
  {Croquet Laboratories} performing the duties of a principal protocol
  engineer and architectural advisor respectively.  My work is mainly
  concerned with the design and implementation of the \texttt{syzygy}
  canonical bridge (Ethereum to Solana), where I do both research and
  development andq designing the blockchain interoperability for the
  \emph{Multisynq} distributed system.  Before this, I was a full-time
  technical lead for the Hyperledger Iroha v2 project, orchestrating
  the assembly of the core development team, decisions on technologies
  and the architecture of the ledger.

  I'm an active participant in
  scientific\cite{sspr-mdpi,stress-test-mdpi} and
  industry\cite{rl-video} conferences.  I have penned scientific
  publications both during my
  studies\cite{charge-frac,cpi}, as well as during
  the free time I had as a full time
  employee\cite{PhysRevD.108.096012,langev,sspr-mdpi,sspr}.
%
}%

\author{Aleksandr Petrosyan}
\date{\today}

% \renewcommand{\myhomephone}
% {%
%   \phone{374}{96}{240350}%
% }%

\renewcommand{\myhomephone}
{%
  \phone{374}{33}{647998}%
}%

%% \renewcommand{\myhomeaddress}
%% {
%%   \address{Dro~st.~\nth{4} bld.}{apt.52}{0069} {Yerevan}{Armenia}
%% }

\renewcommand{\myhomeaddress}
{%
  \address{Dro St.}{\nth{4} Building} {0069}{Yerevan}{Armenia}%
}%

\renewcommand{\myemail}%
{%
  \email{appetrosyan@icloud.com}%
}

\renewcommand{\myGitHub}{%
  \href{https://github.com/appetrosyan}{\faIcon{github}/appetrosyan}%
}


\limitpages{4}
% A C.V. should generally fit on a single double-sided page.

\begin{document}
\maketitle

%   \eachpageornament{41}
\vfill
\section*{Consulting}
\job{Sep. 2023}%
    {Present}%
    {Eclipse Laboratories}%
    {Senior Distributed Systems Consultant\footnote{I am performing
        the duties of a full time founding protocol engineer. }}%
    {%
      I implemented the data availability, and Solana ingest
      capabilities for the Eclipse main network (launching this
      November).  I am currently in charge of the research and
      development of the first full state Merkelisation on a Solana
      network, implemented the upgrade paths from Solana \texttt{1.16}
      to \texttt{1.17} to \texttt{Agave}.  I am working on the
      canonical bridge between Ethereum and our Rollup using Solana.
      Before then I spearheaded the initiatives for the Eclipse test
      network and main network for developers beta launches.  I am
      also the liaison for the collaboration with Anza on the effort
      of the \emph{modular solana virtual machine} project.
    }%

\job{Feb. 2024}%
    {Sep. 2024}%
    {Croquet/Multisynq\footnote{Former Multisync}}%
    {Blockchain Technology Advisor}%
    {%
      I provided detailed explanations and expositions, guiding the
      choice of correct blockchain technology and tokenomic model.  I've
      done preliminary training of Multisynq employees in the
      \href{https://github.com/Greybeard-Entertainment/rust}{Rust
        programming language}, as well as vetted other contractors' work.  As
      part of a demonstration, I implemented a toy tokenomic model based off
      of the Helium program library.
    }%


\vfill
\section*{Employment history} {%
  \setlength{\parindent}{0in}%

  \job{Sep 2024}%
  {Present}%
  {Greybeard Consulting}%
  {CEO}%
  {%
    Founded and worked as part of a consultancy.  My job was to find
    appropriate customers, make public appearances, and co-ordinate
    work with the other members of the consultancy.
  }%

\job{Sep 2021}%
    {Sep 2023}%
    {Soramitsu}%
    {Tech Lead}%
    {%
      I am a senior Rust software engineer, Tech Lead and product owner
      of Hyperledger Iroha~v2.  I assembled the team that is considered
      the best in the company, and produced the first Long-term support
      release candidate of Iroha~v2\footnote{Pre-release but reliable
      enough for testing on production systems}. I built a bug triage
      system that reduced the average lifetime of a bug in Iroha~v2 to
      just one week. I designed our trigger system (with WASM support)
      and smartcontract system, integrated Iroha~v2 with Hyperledger
      Cactus, improved the throughput of Iroha's consensus from 10~TPS
      to 30000~TPS. I architected the ADB CSD/RTGS linkages PoC, results
      presented at the Asia Impact Webinar 2023.  I'm now working on
      integrating Iroha~v2 with Parity Substrate and Polkaswap.  I'm now
      mainly focused on optimising the Iroha ISA, and creating a
      stable-ABI fully programmable blockchain. }%

\job{Oct 2020}%
    {Sep 2021}%
    {PolyChord}%
    {Software engineer}%
    {%
      Worked on \term{supernest}, which accelerates Bayesian inference
      by a factor of 1000. Work was presented at the MaxEnt 2022
      conference in Paris. Follow up research to be presented at MaxEnt
      2023 in Munich.
    }%


\begin{comment}
  \job{Jul 2020}{Oct 2020} {Cambridge DAMTP} {QA intern} {Designed a
    neural network for vetting the applicability of the NHSx X-ray
    data for automated diagnosis of COVID-19, which was comparable in
    precision, but required 5--6 times less processing power.  }
\end{comment}

\begin{comment}
  \job{Jul 2018}{Oct 2018} {Cambridge University Computer Lab}
  {Software Development Intern} {Designed, and improved a dynamic road
    network All-pairs shortest path solver, using a highly
    paralleliseable graph traversal framework, which improved
    performance by a factor of 50.  }
\end{comment}

\begin{comment}
  \job{Jan 2013}{May 2015}  {Masterlagebra.org} {Backend Developer}
  {Designed and implemented problems for university level automated
    linear algebra training program. Problems were related to complex
    numbers, Gram-Schmidt ortho-normalisation process, analytical geometry
    problems involving planes and lines in higher dimensions.
  }
\end{comment}

\begin{comment}
  \section*{Volunteer work} \job{2015}{2017}{Queens' College}
  {Technical director} {I supervised a self-motivated team of
    volunteers to set up and dismantle decorations at the Queens'
    College Fitzpatrick Hall, as well as handle Audio visual equiment
    including but not limited to ROBE colorspots, human sized
    speakers, trusses, stage hydraulics, a laser projector and
    multiple smoke machines.  During QErgs 2016 (which is the largest
    indoors rowing competition in the world), we had a large
    issue. The Ergs (devices which measured the sportspeople's
    performance) were located at the far side of the hall, right under
    the projector screen and opposite the projector. They had to be
    connected to the projector via a CAT6 shielded wire. As it turned
    out, the shielded wire didn't have enough range, and the race
    would have been called off.  I quickly made a skype call with
    desktop sharing to the Wi-Fi connected laptop at the far side. }
\end{comment}

% \pagebreak
\section*{Education}%

\education{2019}%
          {2020}%
          {University of Cambridge}%
          {MSci, \nth{1}}%
          {Physical Natural Sciences Tripos}%
          {%

            Took courses in Relativistic Astrophysics and Cosmology as
            well as Quantum Field theory read by the Department of
            Mathematics. One of the top results in my Research project
            on ``Accelerated Nested Sampling in context of
            Cosmological Parameter estimation''% \cite{sspr}. My
            optimisations lead to an improvement of Nested Sampling's
            performance by a factor of 30.
          }%


\education{2015}%
          {2018}%
          {University of Cambridge}%
          {BA, Upper \nth{2}}%
          {Physical Natural Sciences Tripos}%
          {
            Received the Cambridge Trust scholarship. Formal training
            in Functional Programming and OOP.\@ Took optional courses
            in Theoretical Physics. \nth{1} class in Research Review
            on the topic of ``Charge Fractionalisation in Poly-acetylene''%
            \cite{charge-frac}. I was forced to intermit for a year due to
            mandatory millitary service in my home country of Armenia,
            which resulted in a lower test score.
          }
\begin{comment}
  \education{2013}%
  {2015}%
  {Russian-Armenian (Slavonic) University (RAU)}%
  {BA, GPA 3.82\footnote{Discontinued in favour of Cambridge}}%
  {Electronics and Nanoelectronics}%
  {%
    I was an active participant\cite{cu2o,measurement} in the
    scientific research happening at the RAU solid state
    laboratory.  I was consistently the top student at the
    Institute of Mathematics and High Technology of RAU.  My
    coursework on digital circuits was given the first and to
    this date only perfect score for coursework at RAU.
  }%
\end{comment}
\vfill
\section*{Non-exhasutive list of Projects}%
%
\project{Hyperledger Iroha v2.0}%
        {2022}%
        {https://github.com/hyperledger/iroha}%
        {%
          A Rust rewrite of the popular blockchain ledger based on a custom
          RISC architecture, with vector extensions and a parallel consensus
          algorithm without sharding.
        }%

  \project{Hyperledger Ursa}{2022}{https://github.com/hyperledger/ursa}{
    The one and only cryptographic library used across hyperledger
    projects.
  }

  \project{rust-in-2x}{2024}{https://github.com/Greybeard-Entertainment/rust-in-2x}
  {%
    A work-in-progress book on the Rust programming language.
  }%

  \project{disposable-key}{2023}{https://github.com/Greybeard-Entertainment/disposable-key}
  {%
    Replace the annoying beep in Emacs with something far more useful,
    by creating ephemeral key bindings for un-commonly used functions.
  }%

  \project{analgesic}{2021}{private}{A Rust reimplementation of
    \term{getdist} for analysis of Bayesian inference samples, with an
    emphasis on precision, and performance.
  }

  \project{Rusty-nail}{2021}{https://gitlab.com/a-p-petrosyan/rusty-nail}{%
    A rewrite of Quake 1 Darkplaces engine in Rust with additional
    Vulkan support.
  }

  \project{paranestamol}{2020}{https://github.com/appetrosyan/paranestamol}
  {%
    A Rust + python + QML gui front-end for \term{nestcheck} and \term{anesthetic}
    to fix nested sampler induced headaches.
  }

  \project{superNest}{2020}{https://gitlab.com/a-p-petrosyan/sspr}
  {%
    A python package~\cite{sspr-mdpi} for including \emph{stochastic
      superpositional posterior repartitioning}~\cite{sspr} into your nested
    sampler of choice.
  }%

  \project{anesthetic}{2020} {https://github.com/appetrosyan/anesthetic}
  {%
    A package for visualising and plotting posterior chains for nested
    samplers such as \term{PolyChord}.
  }

  \project{partial-config}{2024}{https://github.com/appetrosyan/partial\_config}
  {%
    A layered configuration parser that provides useful feedback for
    mis-configuration.  Used in \texttt{syzygy}.
  }%
}
\vfill
\section*{Awards}
\setlength{\parindent}{0in}%
\award{2020}{University of Cambridge Amateur composition competition}{\nth{2}
  place}

\award{2015}{Russian Armenian Slavonic University}{Academic
  Excellence}

\award{2014}{Russian Armenian Slavonic University}{Best Course Work}

\award{2013}{\nth{44} International Physics Olympiad}{Bronze Medal}

\award{2013}{\nth{9} International Zhautykov Olympiad}{Bronze Medal}

\award{2012}{\nth{43} International Physics Olympiad}{Honourable
  Mention}

\award{2008}{Annual School informatics Olympiad}{\nth{1} Place}
% \eachpageornament{41}
\vfill
\pagebreak
\section*{Technologies}
\vfill
  \begin{multicols}{4}
    \begin{skills}{Rust}
    \item adv.~\term{serde}
    \item \term{SCALE}
    \item \term{kani}
    \item \term{rayon}
    \item \term{tokio}
    \end{skills}

    \begin{skills}{Blockchain}
    \item \term{WASM}
    \item \term{Substrate}
    \item \term{zkStark}
    \item \term{DEX}
    \end{skills}

    \begin{skills}{Functional}
    \item \term{Haskell}
    \item \term{Rust}\footnote{Although Rust isn't technically a functional language, it supports the functional style very well}
    \item \term{Standard ML}
    \item \term{Scheme}
    \end{skills}

    \begin{skills}{CI/CD}
    \item Github actions
    \item Docker
    \item \term{Allure/pytest}
    \item \term{DCO}
    \item Gitlab CI
    \end{skills}


    \begin{skills}{Data Science}
    \item \term{numpy/pandas/bokeh}
    \item \term{MatLab/Octave}
    \item \term{Tensorflow}
    \item \term{PyPi}
    \end{skills}

    \begin{skills}{HPC}
    \item \term{NUMA}
    \item \term{OpenMPI}
    \item \term{OpenMP}
    \end{skills}

    \begin{skills}{Interpersonal}
    \item Society management\footnote{President of Cambridge Armenian
        Society, Queens' Ents technical director and Queens' Junior
        Combination Regalia Computer officer. }
    \item Negotiation\footnote{With Asian Development Bank ---
        contract on designing a cross-border DVP system; Hyperledger
        foundation ---  marketing Iroha 2}
    \item Teaching
    \end{skills}

    \begin{skills}{OOP}
    \item \term{C++}
    \item \term{Java/Kotlin}
    \item \term{OCaML}
    \end{skills}


    \begin{skills}{Testing}
    \item {\footnotesize Property-based}
    \item {\footnotesize Fuzzing}
    \item {\footnotesize Functional}
    \item {\footnotesize Load testing}
    \item {\footnotesize Test-driven development}
    \end{skills}
\begin{comment}
  \begin{skills}{Other Languages}
  \item \term{Python}
  \item \term{Bash}
  \item \term{Elisp}
  \end{skills}
\end{comment}
\begin{comment}
  \begin{skills}{OS/Containers}
  \item \term{NixOS}
  \item \term{Linux}\footnote{Mainly Arch-based, but with experience
      of deploying Alpine based containers and static linking}
  \item \term{BSD}
  \item \term{Mac OS X}\footnote{With low-level API knowledge for
      mouse tracking, key-logging, SIP circumvention}
  \end{skills}
\end{comment}
  \end{multicols}
\vfill



\bibliography{bibliography}
\vfill
\end{document}
