\documentclass{CurriculumVitae}[10pt, condensed]
\usepackage{multicol}
\usepackage{enumitem}
\usepackage[square,numbers]{natbib}
\bibliographystyle{abbrvnat}
\usepackage{background}
\usepackage{pgfornament}
\usepackage{comment}
\usetikzlibrary{calc}
\usepackage[super]{nth}


\backgroundsetup{
scale=1,
opacity=1,
angle=0,
color=black,
contents={%
\begin{tikzpicture}[every node/.style={inner sep=0pt}]
\node[anchor=north west](CNW)
at (current page.north west) {\pgfornament[width=1.75cm]{61}};
\node[anchor=north east](CNE)
at (current page.north east) {\pgfornament[width=1.75cm,symmetry=v]{61}};
\node[anchor=south west](CSW)
at (current page.south west) {\pgfornament[width=1.75cm,symmetry=h]{61}};
\node[anchor=south east](CSE)
at (current page.south east) {\pgfornament[width=1.75cm,symmetry=c]{61}};
\end{tikzpicture}%
  }
}


% Override this to blank if you don't want to have a coverletter.
\renewcommand{\coverletter} {

  I'm a former Physicist\cite{langev,measurement,cpi,charge-frac,cu2o}
  who works as a Tech Lead at Hyperledger, a subsidiary of the Linux
  Foundation, currently leading the development and design of
  Hyperledger Iroha, the world's first parallel blockchain ledger.
  I'm actively involved in the design of its architecture, as such I'm
  involved with the creation of the runtime system the permission
  validation system of Iroha. We are designing a fully dynamic data
  model based on WASM.

}


\author{Aleksandr Petrosyan} \date{\today}

% \renewcommand{\myhomephone}
% {%
%   \phone{374}{96}{240350}%
% }%


\renewcommand{\myhomephone}
{
  \phone{44}{7864}{647998}
}

% \renewcommand{\myhomeaddress}
% {
%   \address{Dro~st.~\nth{4} bld.}{apt.52}{0069} {Yerevan}{Armenia}
% }

\renewcommand{\myhomeaddress}
{%
  \address{Dro St.}{\nth{4} Building} {0069}{Yerevan}{Armenia}%
}%

\renewcommand{\myemail}%
{%
  \email{appetrosyan@icloud.com}%
}

\renewcommand{\myGitHub}{%
  \href{https://github.com/appetrosyan}{github.com/appetrosyan}
}


\limitpages{4}
% A C.V. should generally fit on a single double-sided page.

\begin{document}
\maketitle


% \newcommand{\eachpageornament}[1]{%
% \begin{tikzpicture}[remember picture, overlay]%
%   \node[anchor=north west, xshift=0.2in, yshift=-0.2in] at (current%
%   page.north west){%
%   \pgfornament[width=2cm]{#1}}; \node[anchor=north east,%
%   xshift=-0.2in, yshift=-0.2in] at (current page.north east){%
%   \pgfornament[width=2cm,symmetry=v]{#1}}; \node[anchor=south%
%   west, xshift=0.2in, yshift=0.2in] at (current page.south west){%
%   \pgfornament[width=2cm,symmetry=h]{#1}}; \node[anchor=south%
%   east, xshift=-0.2in, yshift=0.2in] at (current page.south east){%
%   \pgfornament[width=2cm,symmetry=c]{#1}};%
% \end{tikzpicture}
% }

%   \eachpageornament{41}
\section*{Work Experience} {%
  \setlength{\parindent}{0in}%

  \job{Sep 2021}{Present} {Soramitsu}{Tech Lead} {I am a senior Rust
    software engineer, Tech Lead and product owner of Hyperledger
    Iroha~v2.  I assembled the team that is considered the best in the
    company, and produced the first Long-term support release
    candidate of Iroha~v2\footnote{Pre-release but reliable enough for
      testing on production systems}. I organised a bug triage system
    that reduced the average lifetime of a bug in Iroha~v2 to just one
    week. I designed our trigger system (with WASM support), our WASM
    optimisation pipeline and smartcontract system, integrated
    Iroha~v2 with Hyperledger Cactus, improved the throughput of
    Iroha's consensus from 10~TPS to 30000~TPS. I was the architect of
    the CSD/RTGS linkages PoC (done with Asian Development Bank),
    where I also produced a 60 page report commended for an in-depth
    analysis. I'm now working on integrating Iroha~v2 with Parity
    Substrate and Polkaswap. As part of my work on Iroha I also
    maintain Hyperledger Ursa, a cryptographic library with the
    world's fastest bulletproofs. }


  \job{Oct 2020}{Sep 2021}
  {PolyChord} {Software engineer} { Expanded upon and built the
    package \term{supernest}, which accelerates Bayesian inference by
    a factor of 1000. Worked on Likelihood-free inference, parameter
    speed hierarchy --- functional programming correspondence. Work
    was presented at the MaxEnt 2022 conference in Paris. Follow up
    research to be presented at MaxEnt 2023 in Munich. }
  \job{Jul 2020}{Oct 2020} {Cambridge DAMTP} {QA intern} {Designed a
    neural network for vetting the applicability of the NHSx X-ray
    data for automated diagnosis of COVID-19, which was comparable in
    precision, but required 5--6 times less processing power.  }
  
  \job{Jul 2018}{Oct 2018} {Cambridge University Computer Lab}
  {Software Development Intern} {Designed, and improved a dynamic road
    network All-pairs shortest path solver, using a highly
    paralleliseable graph traversal framework, which improved
    performance by a factor of 50.  }
  
  \job{Jan 2013}{May 2015}  {Masterlagebra.org} {Backend Developer}
  {Designed and implemented problems for university level automated
    linear algebra training program. Problems were related to complex
    numbers, Gram-Schmidt ortho-normalisation process, analytical geometry
    problems involving planes and lines in higher dimensions.
  }
  \section*{Volunteer work} \job{2015}{2017}{Queens' College}
  {Technical director} {I supervised a self-motivated team of
    volunteers to set up and dismantle decorations at the Queens'
    College Fitzpatrick Hall, as well as handle Audio visual equiment
    including but not limited to ROBE colorspots, human sized
    speakers, trusses, stage hydraulics, a laser projector and
    multiple smoke machines.  During QErgs 2016 (which is the largest
    indoors rowing competition in the world), we had a large
    issue. The Ergs (devices which measured the sportspeople's
    performance) were located at the far side of the hall, right under
    the projector screen and opposite the projector. They had to be
    connected to the projector via a CAT6 shielded wire. As it turned
    out, the shielded wire didn't have enough range, and the race
    would have been called off.  I quickly made a skype call with
    desktop sharing to the Wi-Fi connected laptop at the far side. }
  \section*{Education}%

  \education{2019}{2020} {University of Cambridge} {MSci, \nth{1}}
  {Physical Natural Sciences Tripos} {Took courses in Relativistic
    Astrophysics and Cosmology as well as Quantum Field theory read by
    the Department of Mathematics. One of the top results in my
    Research project on ``Accelerated Nested Sampling in context of
    Cosmological Parameter estimation''. My optimisations lead to an
    improvement of Nested Sampling's performance by a factor of 30. }

  \education{2015}{2018}
  {University of Cambridge} {BA, Upper \nth{2}}{Physical Natural Sciences
    Tripos} {Received the Cambridge Trust scholarship. Formal training
    in Functional Programming and OOP.\@ Took optional courses in
    Theoretical Physics. \nth{1} class in Research Review on the topic
    of ``Charge Fractionalisation in Poly-acetylene''. Was forced to
    intermit for a year due to mandatory millitary service in my home
    country of Armenia. }

  \section*{Non-exhasutive list of Projects}%
  % 
  \project{Hyperledger Iroha v2.0}{2022}{https://github.com/hyperledger/iroha}
  {A Rust rewrite of the popular blockchain ledger based on a custom
    RISC architecture, with vector extensions and a parallel consensus
    algorithm without sharding.
  }

  \project{Hyperledger Ursa}{2022}{https://github.com/hyperledger/ursa}{
    The one and only cryptographic library used across hyperledger
    projects. 
  }
    
  \project{analgesic}{2021}{private}{A Rust reimplementation of
    \term{getdist} for analysis of Bayesian inference samples, with an
    emphasis on precision, and performance. }

  \project{Rusty-nail}{2021}{https://gitlab.com/a-p-petrosyan/rusty-nail}{A
    rewrite of Quake 1 Darkplaces engine in Rust with additional
    Vulkan support. }
  
  \project{paranestamol}{2020}{https://github.com/appetrosyan/paranestamol}{A
    Rust + python + QML gui frontend for \term{nestcheck} and \term{anesthetic}
    to fix nested sampler induced headaches.}

  \project{superNest}{2020}{https://gitlab.com/a-p-petrosyan/sspr} {A
    python package~\cite{sspr-maxent} for including \emph{stochastic
      superpositional posterior repartitioning}~\cite{sspr} into your
    nested sampler of choice.
  }

  \project{anesthetic}{2020} {https://github.com/appetrosyan/anesthetic}
  {A package for visualising and plotting posterior chains for nested
    samplers such as \term{PolyChord}.
  }

  \project{qDoList} {2019} {https://github.com/appetrosyan/qDoList}{A
    To-Do list manager based off of QML and \term{Kirigami}.
  }

  \project{ADynamicSP} {2018}
  {https://gitlab.com/DynamicSP/ADynamicSP/tree/master} {A
    \term{Ligra}/\term{Ligra}+ based Shortest path problem solver for
    dynamic networks. }
  \project{Geary} {2015} {https://github.com/GNOME/geary} {A
    \term{Vala}/\term{GTK} based simple email client, conforming to the
    \term{Gnome} \emph{Human Interface Design} guidelines.}

}
\pagebreak
  \section*{Skills and Technologies}
  \setlist{nosep}

  \begin{multicols}{4}
    \begin{itemize}[topsep=0pt]
      \setlength{\itemsep}{-0.3em}
    \item {\footnotesize Version Control}
      \begin{itemize}[topsep=0pt, partopsep=0pt]
        \setlength{\itemsep}{-0.3em}
      \item \term{Git}
      \item \term{Gerrit}
      \item \term{Pijul}
      \item \term{darcs}
      \end{itemize}
      
    \item {\footnotesize CI/CD}
      \begin{itemize}[topsep=0pt, partopsep=0pt]
        \setlength{\itemsep}{-0.3em}
      \item {\footnotesize GitHub actions}
      \item {\footnotesize GitLab CI}
      \end{itemize}
      
    \item {\footnotesize Data Science}
      \begin{itemize}[topsep=0pt, partopsep=0pt]
        \setlength{\itemsep}{-0.3em}
      \item \term{numpy/pandas/bokeh}
      \item \term{MatLab/Octave}
      \item \term{Spigot}
      \item \term{Seaborn}
      \item \term{Tensorflow}
      \item \term{Cython}
      \item \term{PyPi}
      \end{itemize}
      
    \item {\footnotesize HPC}
      \begin{itemize}[topsep=0pt, partopsep=0pt]
        \setlength{\itemsep}{-0.3em}
      \item \term{NUMA}
      \item \term{HPC}
      \item \term{CILK}
      \item \term{OpenMPI}
      \item \term{OpenMP}
      \item \term{MPI}
      \end{itemize}
      
    \item {\footnotesize OOP}
      \begin{itemize}[topsep=0pt, partopsep=0pt]
        \setlength{\itemsep}{-0.3em}
      \item \term{C++}
      \item \term{Java/Kotlin}
      \item \term{OCaML}
      \end{itemize}
      
    \item {\footnotesize Functional Programming}
      \begin{itemize}[topsep=0pt, partopsep=0pt]
        \setlength{\itemsep}{-0.3em}
      \item \term{Haskell}
      \item \term{Rust}
      \item \term{Standard ML}
      \item \term{Scheme}
      \end{itemize}
      
    \item {\footnotesize Testing techniques}
      \begin{itemize}[topsep=0pt, partopsep=0pt]
        \setlength{\itemsep}{-0.3em}
      \item {\footnotesize Property-based}
      \item {\footnotesize Fuzzing}
      \item {\footnotesize Functional}
      \end{itemize}
      
    \item {\footnotesize Programming languages}
      \begin{itemize}[topsep=0pt, partopsep=0pt]
        \setlength{\itemsep}{-0.3em}
      \item \term{Python}
      \item \term{Bash}
      \item \term{Elisp}
      \end{itemize}
      
    \item {\footnotesize OS container proficiency:}
      \begin{itemize}[topsep=0pt, partopsep=0pt]
        \setlength{\itemsep}{-0.3em}
        
      \item \term{NixOS}
      \item \term{Linux}\footnote{Mainly Arch-based, but with experience
          of deploying Alpine based containers and static linking}
      \item \term{BSD}
      \item \term{Mac OS X}\footnote{With low-level API knowledge for
          mouse tracking, key-logging, SIP circumvention}
      \end{itemize}
      
    \item {\footnotesize \LaTeX{}}
      
    \item {\footnotesize Blockchain tech}
      \begin{itemize}[topsep=0pt, partopsep=0pt]
        \setlength{\itemsep}{-0.3em}
      \item \term{WASM}
      \item \term{WAT}
      \item \term{Substrate}
      \item Zero knowledge proof
      \item Nominated Proof of Stake
      \item Proof of space
      \item \term{DEX}
      \end{itemize}
      
    \item {\footnotesize Interpersonal Skills}
      \begin{itemize}[topsep=0pt, partopsep=0pt]
        \setlength{\itemsep}{-0.3em}
      \item society management\footnote{President of Cambridge Armenian
          Society, Queens' Ents technical director and Queens' Junior
          Combination Regalia Computer officer. }
      \item negotiation\footnote{With Asian Development Bank ---
          contract on designing a cross-border DVP system; Hyperledger
          foundation ---  marketing Iroha 2}
      \item teaching
      \end{itemize}
      
    \end{itemize}

  \end{multicols}



\section*{Awards}
\award{2015}{Russian Armenian Slavonic University}{Academic
  Excellence}

\award{2014}{Russian Armenian Slavonic University}{Best Course Work}

\award{2013}{\nth{44} International Physics Olympiad}{Bronze Medal}

\award{2013}{\nth{9} International Zhautykov Olympiad}{Bronze Medal}

\award{2012}{\nth{43} International Physics Olympiad}{Honourable
  Mention}

\award{2008}{Annual School informatics Olympiad}{\nth{1} Place}
% \eachpageornament{41}

\bibliography{bibliography}

\end{document}
