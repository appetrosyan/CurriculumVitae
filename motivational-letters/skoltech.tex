% Created 2020-07-23 Thu 17:01
% Intended LaTeX compiler: pdflatex
\documentclass[a4paper, twocolumn]{article}
\usepackage[utf8]{inputenc}
\usepackage[english]{babel}
\usepackage{hyperref}
\usepackage{ulem}
\usepackage{grffile}
\usepackage{longtable}
\usepackage{capt-of}
\author{Aleksandr Petrosyan}
\date{\today}
\title{Motivational Letter for a PhD application to Skoltech}
\hypersetup{
 pdfauthor={Aleksandr Petrosyan},
 pdftitle={Motivational Letter for a PhD application to Skoltech},
 pdfkeywords={},
 pdfsubject={},
 pdfcreator={Emacs 27.0.91 (Org mode 9.3)}, 
 pdflang={English}}
\begin{document}

\maketitle
\section{Introduction}
\label{sec:org3d65b1f}
In the informational age, we as scientists are more than ever before
dependent on technology. True there are those who would view this is
as an incursion into the traditional approaches of pen paper and
mind, but the world has moved on. Gone are those days, as very few
problems remain on a level where an individual's prowess in algebra
is sufficient to resolve a crisis in a field of Physics. Problems
like the four-colour theorem have a proof so complicated, that not
only is a human incapable of producing it within a lifetime, but
also barely able to audit the proof. Today's physicists all without
exception have to have knowledge of Computer science and advanced
computational methods. This is where my expertise would be
invaluable.
\section{About me.}
\label{sec:org535a41d}

I'm a recent graduate from the University of Cambridge. I've studied
in the Natural Sciences Tripos; a prestigious course that
encapsulates all areas of Physics, starting from Soft to Quantum
condensed matter Physics, and including multiple opportunities to do
independent research, my favourite part of the course. 

I have on multiple occasions had the pleasure of working on rather
complex projects which required a large degree of creativity and
diligence on my part. I had managed to measure the ratio \(e /
  \hbar\) to an accuracy and precision of the order of magnitude of
the top result in the world: \emph{CODATA}, earning me a high first in the
Experimental module. I have conducted theoretical research in
collaboration with many of my supervisors, most of which is due to
be published in September/October of this year, including research
into topic for which I had not been instructed in officially (our
course on statistical mechanics does not include non-equilibrium
processes at the undergraduate level). Finally, I have 7+ years of
programming experience, which I had repeatedly demonstrated in many
computational exercises, including ones from the Computer Science
Tripos. For example, on my first year I had single-handedly
implemented a \emph{Fibonacci heap}. On my third year I designed a
\emph{multi-threaded}, high performance simulation of an \emph{Ising model
magnet in 2D}. 

I have a particular passion for problems that are too difficult to
be solved from first principles: problems like the aforementioned
\emph{four-colour theorem}. As one can see from my academic record, I had
shown the best results when presented with a large, and difficult to
solve problem, and allowed room to think independently.My best work
was the result of countless hours spent trying to solve one big
problem. In particular, my most recent work, on accelerating
Bayesian inference, had been one of the few top scoring research
projects in the year. We had accomplished what several algorithms
could not in the past. We had managed to accelerate Nested Sampling
to a point where a complicated Cosmological analysis that usually
takes \textbf{a week on an HPC cluster} was performed on my (mediocre)
\textbf{home computer within two days}. Not did that not result in a loss
of accuracy, but my optimisation managed to improve both speed and
precision by two orders of magnitude.  An extension of my method
(which I proposed) has the potential to revolutionise model fitting
in cosmology and particle physics, where Bayesian inference is a key
method.

Because of economic factors, during my studies I had to have a
day-job, which in my case was academic instruction. During my brief
tenure as a Tutor I have supervised and instructed approximately 50
students, all of which had gone on to remarkable careers (in fact
one of my former students was in the same class as I was, because of
the fact that I had started studying at RAU before Cambridge). I
have had multiple internships, primarily on academic computational
projects: currently I'm working on a machine-learning-based
algorithm for sorting and quality checking chest X-ray images and CT
scans of COVID-19 patients. A year ago, I had been working on
\emph{ligra} in application to dynamical road network optimisation
problems, and agent-based models for the city of San Francisco. As
such I had managed to make \texttt{ligra} approximately ten times faster. 

\section{How I fit into the project}
\label{sec:org695b4b7}

As you well know, the anomalous diffusion project involves a
significant amount of programming, and in particular one related to
neural networks and machine learning, of which I'm a specialist. I
have on multiple occasions investigated the performance and
applicability of machine learning models, as well as worked on
optimising said models to maximum utility.

The results of my research will also need to be applied and sanity
checked. While co-authoring the ``Relativistic Langevin equation''
with Prof. A. Zaccone, I had broadened immensely my knowledge of
non-equilibrium processes, of which anomalous diffusion is one. I
will be able to identify, if my neural network mispredicts the
paths, and whether or not a purely image interpolation method would
even be valid\footnote{Nvidia's DLSS actually uses a hybrid dynamical and
static upscaling neural network on their GPUs.}. 

Moreover, this work is rather involved; it requires diligence,
discipline and unwavering enthusiasm in the face of tedium. These
are the qualities that I obtained at Cambridge, during the long
hours of exam preparation and the fast-paced practical classes. I
had learned how to work on a problem before it crumbles away, but
also to do so quickly and efficiently.

Finally, the results of all of this research to be collated and
written in a form suitable for academic readers all throughout the
world. This is where my near-native proficiency in English and
particular experience of report and article authoring shall be
invaluable. We had been writing up a lab report almost every two
months, which had to be a publication quality paper. I am so well
attuned to \LaTeX{} that it's genuinely easier for me to write-up my CV
and motivation letter in it, rather than use a word processor.

For me, the particular position at Skoltech is an opportunity to
breach academia and begin a scientific career, by studying at a
world-class institution. Skolkovo is of particular interest to me,
as it is closer to my home, both in terms of mentality and
geographically. 



\section{Concluding remarks}
\label{sec:org575b53b}

I hope that I have convinced the admissions committee of my
qualifications, based on both my academic work, my abilities,
accomplishments and experience. I sincerely hope that we will have a
fruitful collaboration. 
\end{document}