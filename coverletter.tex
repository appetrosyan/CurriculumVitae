% Created 2023-02-15 Wed 18:07
\documentclass[11pt]{scrlttr2}
\usepackage[utf8]{inputenc}
\usepackage[T1]{fontenc}
\usepackage{fontsize}
\usepackage{anyfontsize}
\usepackage{graphicx}
\usepackage{longtable}
\usepackage{wrapfig}
\usepackage{rotating}
\usepackage[normalem]{ulem}
\usepackage{amsmath}
\usepackage{amssymb}
\usepackage{capt-of}
\usepackage{hyperref}
\usepackage[skip=10pt,indent=5pt]{parskip}
\usepackage[a4paper, total={7in, 9in}]{geometry}

\usepackage{ebgaramond}
\usepackage{lato}
% \usepackage[letterspace=50]{microtype}
\usepackage{fontawesome5}
\changefontsize[17pt]{12.5pt}

\setkomavar{fromname}{Aleksandr Petrosyan} % Your name
\setkomavar{signature}{Aleksandr Petrosyan, Iroha 2 Tech Lead} 

\newkomavar{subtitle}\setkomavar{subtitle}{Iroha 2 Technical Lead} 

\setkomavar{fromaddress}{Dro st. 4-th building, Apartment 57, Yerevan 0069, Armenia}
\setkomavar{fromphone}{+374-33-647-998} % Your phone number
\setkomavar{fromemail}{appetrosyan@icloud.com} % Your email address
\setkomavar{place}{Yerevan, Armenia} 
\setkomavar{date}{\today} 
\setkomavar{subject}{Cover Letter to Eclipse} 

%\setkomavar{fromfax}{+1 (123) 456 7890} % Your fax number
%\setkomavar{fromurl}{https://www.LaTeXTemplates.com} % Your personal website
\setkomavar{firsthead}{ % Style the first page header
	\centering\scshape % Text position and styling
	{\huge\textls{\usekomavar{fromname}}}\\[4pt] % Your name
	{\large\usekomavar{subtitle}} % Subtitle
}

\setplength{firstheadvpos}{0.045\paperheight} 
\setkomavar{firstfoot}{
	\centering\scshape\lsstyle
	{\renewcommand{\\}{~{\large\textperiodcentered}~}%
      \usekomavar{fromaddress}}\\
	{\small{\footnotesize\faEnvelope}%
      \hspace{4pt}\usekomavar{fromemail}%
      \hspace{8pt}{\footnotesize\faMobile*}%
      \hspace{4pt}\usekomavar{fromphone}} 
}
\setplength{firstfootvpos}{0.92\paperheight}

\setlength{\parindent}{0pt}
\author{Aleksandr Petrosyan}
\date{\today}
\title{About me}
\hypersetup{
 pdfauthor={Aleksandr Petrosyan},
 pdftitle={About me},
 pdfkeywords={},
 pdfsubject={},
 pdfcreator={Emacs 28.2 (Org mode 9.5.5)}, 
 pdflang={English}}
\begin{document}
\begin{letter}{
	% Name and address of the person to whom the letter is being sent
}

\opening{Dear Eclipse}

I am a Physicist turned Software Architect working on the Hyperledger
Iroha distributed ledger.  I've been working on a combination of
consensus, protocol Instruction set architecture and zero-knowledge
problems and in this document I'd like to provide you with a guided
tour of my career and talk about a few of the more recent aspects of
Iroha that I worked on.

Firstly, after being hired in 2021, I've been swiftly promoted to
technical lead in 2022. This greatly expanded the roster of my
responsibilities. While I still solve engineering tasks on occasion, I
now primarily focus on research, management, mentorship and
communication between projects. I up-sized the Iroha team from three
engineers to 12, oversaw the establishment of synchronisation
processes between projects using Iroha, namely the four SDKs. From
that point onward had a steady stream of monthly releases, reduced the
number of in-production API-breakages to zero, established a cadence
and roadmap for the project and, finally, produced our first LTS
release.

As an architect, I was mainly responsible for prototyping, researching
and implementing the key features of our ledger. Namely, the
smartcontract ecosystem, with triggers, executable WASM support,
optimisations and mitigations for any technical limitations.  I was
also responsible for identifying performance bottlenecks. As an
example, moving away from an in-house actor implementation resulted in
a speedup from 10 TPS to 10 000 TPS. I did not execute on all of the
implementations, however, because that would result in a bottleneck
and potentially put time of the dozen engineers under my supervision
to waste.  As these engineers were of middle to junior levels (due to
budget constraints) it was also my responsibility to break the tasks
down into manageable chunks, and assign them such that they maximised
the productivity and educational value.

During my brief involvement in Hyperledger Ursa, I gained a cursory
understanding of the zero knowledge infrastructure and methods of its
implementation using Rust and integrated circuitry.  While I am no
cryptographer, I can make an informed decision on whether to use
\texttt{Snarks}, \texttt{Starks} or \texttt{bulletproofs}.

My expertise in Rust is a little hard to explain.  I've been doing
competitive programming up to 2015 and afterwards scientific research
programming all the way until 2021, with a mix of C++, Python and
StandardML.  As a consequence, when I decided to change direction and
take up Rust, the transition was extremely fast; I got my first job
and was nominated Senior Rust engineer two weeks after starting to
officially learn the language.  Prior exposure to ownership in Qt,
templated programming in C++, Procedural macros in Emacs lisp,
Haskell's type system and \texttt{async} programming in Python 3.6
meant that I was mainly focused on learning the tooling, and mapping
C++'s implicit lifetime models onto Rust's explicit lifetime paradigm.
I have a good background in compiler construction and exposure to
automatic theorem proving (using \texttt{Isobel}).  As such, I was
involved in hiring decisions for every engineering position involving
Rust, Python, Kotlin, Java, C++ and Swift throughout the company.

As a final note, I have a rich history of participation in the
marketing and presentation aspects of the projects that I am working
on.  I frequently co-write documents presented to potential clients
and users of Iroha.  I am frequently attending conferences as a
speaker, this year, for example, I was presenting at MaxEnt 2023, and
will present at EuroRust 2023.  I had been the coordinator for our
company's team participating in the CSD/RTGS Proof-of-concept in
conjunction with the Asian development bank.  In addition to producing
the code and presenting the successful execution of the PoC, I also
authored a 60~page report on our conclusions and took part in the
panel discussion at the Asia Impact Webinar alongside blockchain
vendors such as Consensys and R3. 

As a closing statement, I'd like to provide you with references:
Ry~Jones(\href{mailto:ry@linux.com}{ry@linux.com}) is the senior
architect at Hyperledger, and the person I've had the most intensive
interactions with. Dr~William Handley
(\href{mailto:wh260@cam.ac.uk}{wh260@cam.ac.uk}) is my academic
supervisor and frequent collaborator.

Sincerely, \\
\usekomavar{signature}
\end{letter}
\end{document}