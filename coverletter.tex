% Created 2023-02-15 Wed 18:07
\documentclass[11pt]{scrlttr2}
\usepackage[utf8]{inputenc}
\usepackage[T1]{fontenc}
\usepackage{fontsize}
\usepackage{anyfontsize}
\usepackage{graphicx}
\usepackage{longtable}
\usepackage{wrapfig}
\usepackage{rotating}
\usepackage[normalem]{ulem}
\usepackage{amsmath}
\usepackage{amssymb}
\usepackage{capt-of}
\usepackage{hyperref}
\usepackage[skip=10pt,indent=5pt]{parskip}
\usepackage[a4paper, total={7in, 9in}]{geometry}

\usepackage{ebgaramond}
\usepackage{lato}
% \usepackage[letterspace=50]{microtype}
\usepackage{fontawesome5}
\changefontsize[17pt]{12.5pt}

\setkomavar{fromname}{Aleksandr Petrosyan} % Your name
\setkomavar{signature}{Aleksandr Petrosyan, Iroha 2 Tech Lead} 

\newkomavar{subtitle}\setkomavar{subtitle}{Iroha 2 Technical Lead} 

\setkomavar{fromaddress}{Dro st. 4-th building, Apartment 57, Yerevan 0069, Armenia}
\setkomavar{fromphone}{+374-33-647-998} % Your phone number
\setkomavar{fromemail}{appetrosyan@icloud.com} % Your email address
\setkomavar{place}{Yerevan, Armenia} 
\setkomavar{date}{\today} 
\setkomavar{subject}{Cover Letter to Ontropy} 

%\setkomavar{fromfax}{+1 (123) 456 7890} % Your fax number
%\setkomavar{fromurl}{https://www.LaTeXTemplates.com} % Your personal website
\setkomavar{firsthead}{ % Style the first page header
	\centering\scshape % Text position and styling
	{\huge\textls{\usekomavar{fromname}}}\\[4pt] % Your name
	{\large\usekomavar{subtitle}} % Subtitle
}

\setplength{firstheadvpos}{0.045\paperheight} 
\setkomavar{firstfoot}{
	\centering\scshape\lsstyle
	{\renewcommand{\\}{~{\large\textperiodcentered}~}%
      \usekomavar{fromaddress}}\\
	{\small{\footnotesize\faEnvelope}%
      \hspace{4pt}\usekomavar{fromemail}%
      \hspace{8pt}{\footnotesize\faMobile*}%
      \hspace{4pt}\usekomavar{fromphone}} 
}
\setplength{firstfootvpos}{0.92\paperheight}

\setlength{\parindent}{0pt}
\author{Aleksandr Petrosyan}
\date{\today}
\title{About me}
\hypersetup{
 pdfauthor={Aleksandr Petrosyan},
 pdftitle={About me},
 pdfkeywords={},
 pdfsubject={},
 pdfcreator={Emacs 28.2 (Org mode 9.5.5)}, 
 pdflang={English}}
\begin{document}
\begin{letter}{
	% Name and address of the person to whom the letter is being sent
}

\opening{Dear Ontropy}

I typicallly send a covering letter to draw attention to some aspects
of my career that don't exactly fit into the C. V. format, as well as
things that might not fit into the natural flow of conversation during
the first interview. I have predominantly an academic background. I've
15 years of C++ experience under my belt: 7 competitive and 7 in a
research capacity. These 15 years include problem solving in a diverse
expanse of fields with one common thread being efficiency and
accuracy. I will not go much into the details of competitive problem
solving, other than it made my University Education tuition-free, and
with good marks. The research is far more interesting to look at,
because real-world problems are amenable to certain kinds of
outside-the-box thinking that I'm known for.

During tenure at the Cambridge University Computer lab, I was working
on an incremental graph traversal problem, where I was essentially
supposed to solve an all-pairs shortest path problem on a graph with
deterministically varying weights. A preliminary human-sale
statistical analysis revealed quite a few heuristics which improved
the performance of my solution to be 10 times more efficient than the
predicted maximum. What is particularly surprising is that the common
wisdom would suggest using a Fibonacci-heap backed Dijksra's
traversal, but the essence of the changing weights meant that the
normally asymptotically efficient solution was sub-optimal and instead
the much more easily parallelisable Bellman-Ford turned out to be
faster.

During the time after graduation from Cambridge I've produced 3
publications on Nested sampling, which are all based off of an idea of
stochastic superpositional prior mixing. Without getting into too much
detail about Bayesian inference, \texttt{supernest} allows inferences
which used to require a supercomputer crunching numbers for a week to
be done in 36 hours on a laptop. This one discovery got me a seat at
both the MaxEnt 2022 and MaxEnt 2023 conferences (as an invited
speaker), and may eventually end up winning John Skilling a Nobel
Prize (yes, \emph{that} John Skilling). I'd be happy to go into
technical details of the inner workings of \texttt{supernest} during
an in-person conversation during the interview. 

Now at this point, you might be wondering why have I chosen to move
away from academia. There are multiple reasons, the one I'd like to
focus on is the genuine intrigue I espouse towards blockchains as a
technology. I have been a firm believer in Free and Open Source
Software for most of my conscious life, and placed great value in
decentralisation and censorship resistance. Blockchains allow for more
people to achieve financial (and other kinds of) independence. It
allows creating foundamentally fair economic systems and perfect
institutional oversight as well as complete anonymity depending on
application. As such, it should not be surprising to you that when I
was looking for my first job in industry, it was in the sphere of
blockchains.

I have both a fairly recent and profoundly deep connection to
Rust. The language itself I'm only using for the past three years in
an official capacity. I learned the language by porting Quake, which
unfortunately got cut short by my first job in industry. Almost two
weeks after learning the ropes of the language, I was offered a senior
position at Soramitsu. How, you might ask? Rust is C++ with elements
of OCaml, which I knew very well from university and competitive
programming, hence how I was able to scale the ranks so quickly. The
trend of meteoric rise continued as three months afterwards I was
offered the position of technical lead for the Hyperledger Iroha~v2
project. I was the first choice among four other senior engineers, due
to a combination of hard skills and management ability on my side.
During my ongoing tenure as Technical Lead I hired fourteen members
into the Iroha~2 core team, which is considered the best in the
company. I've worked on the CSD/RTGS linkages proof of concept, for
which I've received a personal commendation. I am the representative
of the Iroha project and Soramitsu in general to the Hyperledger
Technical Oversight Comiittee. I've been nominated to become a voting
member at the TOC, in addition to being involved with the Hyperledger
Ursa, a cryptographic library which has sadly been retired
recently\footnote{Although if we're being completely fair, the current
  state of affairs is far more advantageous to Soramitsu}.

Owing to my written (and spoken) communication skills I'm often the
face of the company when selling the product to a new customer. My
voice has appeared on many of the adverts circulated espousing the
benefits of Hyperledger Iroha~v2 as compared to competing
technologies, namely Parity Substrate and Iroha~v1. As such I'm well
regarded in both technical and marketing circles.

As a closing statement, I'd like to provide you with references:
Dr~William Handley (\href{mailto:wh260@cam.ac.uk}{wh260@cam.ac.uk}) is
my academic supervisor and frequent
collaborator. Ry~Jones(\href{mailto:ry@linux.com}{ry@linux.com}) is
the senior architect at Hyperledger, and the person I've had the most
intensive interactions with.

Sincerely, \\
\usekomavar{signature}
\end{letter}
\end{document}