% Created 2023-02-15 Wed 18:07
\documentclass[11pt]{scrlttr2}
\usepackage[utf8]{inputenc}
\usepackage[T1]{fontenc}
\usepackage{fontsize}
\usepackage{anyfontsize}
\usepackage{graphicx}
\usepackage{longtable}
\usepackage{wrapfig}
\usepackage{rotating}
\usepackage[normalem]{ulem}
\usepackage{amsmath}
\usepackage{amssymb}
\usepackage{capt-of}
\usepackage{hyperref}
\usepackage[skip=10pt,indent=5pt]{parskip}
\usepackage[a4paper, total={7in, 9in}]{geometry}

\usepackage{ebgaramond}
\usepackage{lato}
% \usepackage[letterspace=50]{microtype}
\usepackage{fontawesome5}
\changefontsize[17pt]{12.5pt}

\setkomavar{fromname}{Aleksandr Petrosyan} % Your name
\setkomavar{signature}{Aleksandr Petrosyan, Eclipse Laboratories Principal Protocol Engineer}

\newkomavar{subtitle}\setkomavar{subtitle}{Eclipse Laboratories Principal Protocol Engineer}

\setkomavar{fromaddress}{Dro st. 4-th building, Apartment 57, Yerevan 0069, Armenia}
\setkomavar{fromphone}{+44-7864-647-998} % Your phone number
\setkomavar{fromemail}{appetrosyan@icloud.com} % Your email address
\setkomavar{place}{Yerevan, Armenia}
\setkomavar{date}{\today}
\setkomavar{subject}{Cover Letter to Hyperlane}

%\setkomavar{fromfax}{+1 (123) 456 7890} % Your fax number
%\setkomavar{fromurl}{https://www.LaTeXTemplates.com} % Your personal website
\setkomavar{firsthead}{ % Style the first page header
    \centering\scshape % Text position and styling
    {\huge\textls{\usekomavar{fromname}}}\\[4pt] % Your name
    {\large\usekomavar{subtitle}} % Subtitle
}

\setplength{firstheadvpos}{0.045\paperheight}
\setkomavar{firstfoot}{
    \centering\scshape\lsstyle
    {\renewcommand{\\}{~{\large\textperiodcentered}~}%
      \usekomavar{fromaddress}}\\
    {\small{\footnotesize\faEnvelope}%
      \hspace{4pt}\usekomavar{fromemail}%
      \hspace{8pt}{\footnotesize\faMobile*}%
      \hspace{4pt}\usekomavar{fromphone}}
}
\setplength{firstfootvpos}{0.92\paperheight}

\setlength{\parindent}{0pt}
\author{Aleksandr Petrosyan}
\date{\today}
\title{About me}
\hypersetup{
 pdfauthor={Aleksandr Petrosyan},
 pdftitle={About me},
 pdfkeywords={},
 pdfsubject={},
 pdfcreator={Emacs 28.2 (Org mode 9.5.5)},
 pdflang={English}}
\begin{document}
\begin{letter}{
    % Name and address of the person to whom the letter is being sent
}

\opening{Dear Hyperlane}

I am a Physicist turned Software Architect who worked on the
Hyperledger Iroha distributed ledger and the Eclipse canonical bridge
``syzygy''.  I've been working on a combination of consensus, protocol
Instruction set architecture and zero-knowledge problems and in this
document I'd like to provide you with a guided tour of my career and
talk about a few of the more recent aspects of Iroha that I worked on.

Firstly, I shall begin with my histoy at Soramitsu, as this is a
closer point of comparison to the position of technical lead, than the
work that I currently do for Eclipse.  After being hired in 2021, I've
been swiftly promoted to technical lead in 2022. This greatly expanded
the roster of my responsibilities. While I still solve engineering
tasks on occasion, I now primarily focus on research, management,
mentorship and communication between projects. I up-sized the Iroha
team from three engineers to 12, oversaw the establishment of
synchronisation processes between projects using Iroha, namely the
four SDKs. From that point onward had a steady stream of monthly
releases, reduced the number of in-production API-breakages to zero,
established a cadence and roadmap for the project and, finally,
produced our first LTS release.

As an architect, I was mainly responsible for prototyping, researching
and implementing the key features of our ledger. Namely, the
smartcontract ecosystem, with triggers, executable WASM support,
optimisations and mitigations for any technical limitations.  I was
also responsible for identifying performance bottlenecks. As an
example, moving away from an in-house actor implementation resulted in
a speedup from 10 TPS to 10 000 TPS. I did not execute on all of the
implementations, however, because that would result in a bottleneck
and potentially put time of the dozen engineers under my supervision
to waste.  As these engineers were of middle to junior levels (due to
budget constraints) it was also my responsibility to break the tasks
down into manageable chunks, and assign them such that they maximised
the productivity and educational value.

At Eclipse I was involved in Architectural decisions and research from
the very start.  Though I am technically a consultant, my
responsibilities lied on the intersection of R&D and Protocol
Engineering, with a pinch of DevOps.  I was to design the fault proof
protocol in collaboration with other engineers.  After finalising the
design, I implemented a part of the canonical bridge, shifting focus
to helping other engineers in my team.  I kept coordinating with
DevOps and during the launch week, was able to fix several major
issues in the canonical bridge.  As a result, despite an incredibly
optimistic deadline, the canonical bridge was launched on time for the
test network, and with no failures recorded to date.  In parallel I
worked on implementing an in-house fork of Solana, which contained an
instruction for BLS12-381 native sig-verify.

At Eclipse the most important contribution to the business aspect was
a comprehensive document on the comparison of various modular
blockchain architectures.  This is where I came across Hyperlane, and
your IBC solution.  I wrote up the equivalent of a 50 page document
providing a relatively detailed risk-benefit analysis of the so-called
Celestium architecture, the usage of the Celestia Blobstream (formerly
Quantum gravity bridge), which was accepted and my responsibility to
implement for the canonical bridge.  I've regularly advised Neel
Somani (the CEO of Eclipse) on matters pertaining to the project,
including the project's internal name: ``syzygy''.

As a closing statement, I'd like to provide you with references:
Ry~Jones(\href{mailto:ry@linux.com}{ry@linux.com}) is the senior
architect at Hyperledger, and the person I've had the most intensive
interactions with. Dr~William Handley
(\href{mailto:wh260@cam.ac.uk}{wh260@cam.ac.uk}) is my academic
supervisor and frequent collaborator.

Sincerely, \\
\usekomavar{signature}
\end{letter}
\end{document}
